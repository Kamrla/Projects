
\documentclass[a4paper,12pt]{article}

%\usepackage{natbib}
\usepackage{amsthm}
\usepackage{amsfonts}
\usepackage{amssymb}
\usepackage{amsmath}
\usepackage{latexsym}
\usepackage{graphicx}
\usepackage{blindtext}
\usepackage{wrapfig}
\usepackage{doc}

\newtheorem*{theorem}{Theorem}
\theoremstyle{definition}
\newtheorem*{definition}{Definition}

\hoffset -1in \topmargin 0mm \voffset 0mm \headheight 0mm
\headsep0mm
\oddsidemargin  20mm     %   Left margin on odd-numbered pages.
\evensidemargin 20mm     %   Left margin on even-numbered pages.
\textwidth   170mm       %   Width of text line.
\textheight  252mm

\makeatletter
\renewcommand\@openbib@code{%
     \advance\leftmargin  \z@ %\bibindent
      \itemindent \z@
     % Move bibitems close together
     \parsep -0.8ex
     }
\makeatother

\makeatletter
\renewcommand\section{\@startsection {section}{1}{\z@}%
                                   {-3.5ex \@plus -1ex \@minus -.2ex}%
                                   {1.5ex \@plus.2ex}%
                                   {\large\bfseries}}
\makeatother

\makeatletter
\renewcommand\subsection{\@startsection {subsection}{1}{\z@}%
                                   {-3.5ex \@plus -1ex \@minus -.2ex}%
                                   {1.5ex \@plus.2ex}%
                                   {\normalsize\bfseries}}
\makeatother

\makeatletter
	\setlength{\abovecaptionskip}{3pt}   % 0.25cm 
	\setlength{\belowcaptionskip}{3pt}   % 0.25cm 
\makeatother

\begin{document}
\pagestyle{empty}

\begin{center}
{\bf \Large TITLE OF SEMINAR PAPER}
\end{center}

\smallskip
\begin{center}
{\large Jan Kamrla}
\end{center}

\smallskip
\begin{center}
Fakulta strojního inženýrství VUT v Brně\\
Ústav automatizace a informatiky\\
Technická 2896/2, Brno 616 69, Česká republika\\
213005@vutbr.cz\\
\end{center}

\bigskip
\noindent Abstract: \textit{This is a sample of the format of your
full paper. For text in abstract and keywords, use Italics, 10pt. Leave one blank line after the Abstract. yay}

\vspace*{10pt} \noindent Keywords: \textit{Write your keywords (6--10 words).
Leave a double blank line after your keywords.}

\bigskip
\section{Úvod}
Ačkoli se lékařští roboti za posledních několik desetiletí ohromně vyvinuli, a jejich
vliv na zlepšení celkové péče o pacienty je nepopiratelný, současný stav robotiky v lekářstvý je pouze na začátku svého vývoje.

Obor lékařské robotiky je mnohem širší, než si lidé obvykle myslí. Ve většině případů se věří, že robotika je pouze pro chirurgy, zubaře, oftalmology a používá se většinou jen v nemocnicích, oépak je ale pravdou. Robotika slouží také lidem, postižení nebo na nich jinak závislí ostatní a kteří nemohli žít normální život. Používá se také pro rentgen, socializaci, rehabilitace a interakce s lidmi nejen v nemocnicích, ale i v domácnostech.

Co se týče chirurgických operací, některé z nich by dnes v lékařství ani nebyly možné kdyby
roboti nebyli k dispozici a tak dobře technologicky prokročilý. Dnes se s jejich pomocí provádějí operace jejichž velká přesnost, kvalita, obratnost a bezpečnost vede k rychlejšímu zotavení pacienta, kratšímu pobytu v nemocnici, snížení krevních ztrát, minimálnímu zjizvení, sníženínému rizika infekce a nižším celkovým nákladům.Lékařští roboti také umožňují pacientům komunikovat s lékaři a specialisty, kteří pomocí robotické pomoci můžou provádět lékařská vyšetření v domácnosti.

\pagebreak

\section{Historie robotiky v lékařství}
\label{sec:2}
Moore, Eric J.. "robotic surgery". Encyclopedia Britannica, 23 Nov. 2018, https://www.britannica.com/science/robotic-surgery. Accessed 6 March 2022.
\\

Konceptem vzdálené chirurgie neboli telechirurgie se v 70. letech zabýval americký Národní úřad pro letectví a vesmír (NASA), který se zajímal o jeho aplikaci pro astronauty na oběžné dráze. Základní myšlenkou bylo, že stroj vybavený chirurgickými nástroji by mohl být umístěn na vesmírné stanici a řízen chirurgem na Zemi. Podobným plánem se zabývala americká Agentura pro pokročilé obranné výzkumné projekty (DARPA). Výzkumníci DARPA pracovali na vývoji vzdálené telechirurgické jednotky, která by umožnila provádět chirurgické zákroky na raněných na bojišti. Ačkoli ani jedna z těchto myšlenek nebyla plně realizována, pokrok v robotických telechirurgických konceptech a v telekomunikačních technologiích umožnil v roce 2001 operaci Lindbergh, při níž francouzský lékař Jacques Marescaux a chirurg kanadského původu Michel Gagner provedli vzdálenou cholecystektomii (odstranění žlučníku) z New Yorku na pacientovi ve francouzském Štrasburku. Navzdory průlomu se telechirurgii nepodařilo získat širokou popularitu z mnoha důvodů, včetně časových zpoždění mezi kontrolním a provozním koncem.

Dalším cílem robotické chirurgie bylo odstranění nežádoucího pohybu. První chirurgický robot, PUMA 560, byl použit v roce 1985 při stereotaxické operaci, při které byla použita počítačová tomografie k vedení robota, když zaváděl jehlu do mozku pro biopsii, což je postup, který dříve podléhal chybám způsobeným třesem ruky během umístění jehly.

\begin{figure}[h]
\begin{center}
\includegraphics[scale=0.25]{Puma 560.png}
\caption{Please write your figure caption here}
\label{fig:1}
\end{center}
\end{figure}


\subsection{Subsection}
\label{subsec:1}
When including a subsection you must use, for its heading, small
letters, 10pt, left justified bold as here.
Use the standard \verb|equation| environment to typeset your equations, however, for multiline equations we recommend to use the \verb|eqnarray| environment (\LaTeX{} users).

\begin{definition}
Let $H$ be a subgroup of a group~$G$.  A \emph{left coset}
of $H$ in $G$ is a subset of $G$ that is of the form $xH$,
where $x \in G$ and $xH = \{ xh : h \in H \}$.
Similarly a \emph{right coset} of $H$ in $G$ is a subset
of $G$ that is of the form $Hx$, where
$Hx = \{ hx : h \in H \}$
\end{definition}

\begin{theorem}
This is a theorem content. Theorem text goes here. 
\end{theorem}

\begin{proof}
Let $z$ be some element of $xH \cap yH$.  Then $z = xa$
for some $a \in H$, and $z = yb$ for some $b \in H$.
If $h$ is any element of $H$ then $ah \in H$ and
$a^{-1}h \in H$, since $H$ is a subgroup of $G$.
However, $zh = x(ah)$ and $xh = z(a^{-1}h)$ for all $h \in H$.
Therefore $zH \subset xH$ and $xH \subset zH$, and thus
$xH = zH$.  Similarly $yH = zH$, and thus $xH = yH$,
as required.
\end{proof}

\section{Problem Solution}
\pagenumbering{roman}
 
% Start numbering with page 2
\setcounter{page}{2}

Figures\footnote{If you copy text passages, figures, or tables from other works, you must obtain \textit{permission} from the copyright holder (usually the original publisher or author). Please enclose the signed permission with the manuscript.} and tables should be numbered as follows: Fig.~1,
Fig.~2,\,\dots{} etc. (see Fig.~\ref{fig:1}), Table 1, Table 2,\,\dots{} etc. (see Table~\ref{tab:1}). Figure caption must be placed below the figure and table caption must be placed above the table. Some reference \cite{SICILIANO_rhandbook}.

%
% For figures use (also *.eps, *.tiff)
%
\begin{figure}[h]
\begin{center}
\includegraphics[scale=0.25]{Puma 560.png}
\caption{Please write your figure caption here}
\label{fig:1}
\end{center}
\end{figure}


%
% For tables use
%
\begin{table}[h] 
\begin{center}
\caption{Please write your table caption here} 
\label{tab:1}
\begin{tabular}{lll}
\hline\noalign{\smallskip}
Parameter & Symbol & Value\\
\noalign{\smallskip}
\hline\noalign{\smallskip}
Param no. 1 & $\delta$ & 0\\
Param no. 2 & $\pi$    & 3.14\\
\hline
\end{tabular}
\end{center}
\end{table}

\section{Conclusion}
\label{sec:2}
%\blindtext
% References
%
\begingroup
\makeatletter
\renewcommand\section{\@startsection {section}{1}{\z@}%
                                   {-3.5ex \@plus -1ex \@minus -.2ex}%
                                   {4.5ex \@plus.2ex}%
                                   {\large\bfseries}}
\makeatother


\bibliography{references}{}
\bibliographystyle{acm}
\endgroup

\end{document}
© 2022 GitHub, Inc.
Terms
Privacy
Security
Status
Docs
Contact GitHub
Pricing
API
Training
Blog
About
